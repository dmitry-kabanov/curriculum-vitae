%------------------------------------------------------------------------------
% Resume of Dmitry Kabanov.
% Created using LaTeX package `moderncv`.
% ------------------------------------------------------------------------------

%----------------------------------------------------------------------------------------
%	PACKAGES AND OTHER DOCUMENT CONFIGURATIONS
%----------------------------------------------------------------------------------------
% Font sizes: 10, 11, 12
% Paper: a4paper, letterpaper, a5paper, legalpaper, executivepaper, landscape
% Font: sans or roman
% Color: nocolor
% Draft/final option: draft, final
% Default options are: a4paper, 11pt, color, final
\documentclass[11pt,a4paper,sans]{moderncv} 

% CV theme: casual (default), classic, oldstyle, banking
\moderncvstyle{classic}
% CV color: blue (default), orange, green, red, purple, grey, black, burgundy
\moderncvcolor{deepblue}

% Width of the left column.
\setlength{\hintscolumnwidth}{2.75cm}

% Font settings.
% Character encoding.
\usepackage[utf8]{inputenc}
% `Merriweather` is a modern screen-friendly font.
\usepackage{merriweather}

% Page geometry.
\usepackage[scale=0.80]{geometry} % Reduce document margins
%\setlength{\hintscolumnwidth}{3cm} % Uncomment to change the width of the dates column
%\setlength{\makecvtitlenamewidth}{10cm} % For the 'classic' style, uncomment to adjust the width of the space allocated to your name
\recomputelengths

% Remove dots in the end of \cventry lines.
\usepackage{xpatch}
\xpatchcmd{\cventry}{.\strut}{\strut}{}{}

\makeatletter
\makeatother

% Personal data.
% ------------------------------------------------------------------------------
%	NAME AND CONTACT INFORMATION SECTION
% ------------------------------------------------------------------------------

\firstname{Dmitry} % Your first name
\familyname{Kabanov} % Your last name

% All information in this block is optional, comment out any lines you don't need
% \title{Curriculum Vitae}

% \address{street and number}{postcode city}{country}
% optional, remove / comment the line if not wanted;
% the "postcode city" and "country" arguments can be omitted or provided empty.
\address{Aachen, Germany}{}{}

\phone[mobile]{+49~(1517)~087-66-15}
\email{kabanov.dmitry@gmail.com}
\social[linkedin]{dmitry-kabanov}
\social[github]{dmitry-kabanov}

% \photo[picture-height][frame-thickness]{photo-filename}
\photo[70pt][0.0pt]{photo}

% ------------------------------------------------------------------------------
% Bibliography settings.
\usepackage[
style=numeric-comp,
bibstyle=numeric,
backend=biber,
giveninits=true,
defernumbers=true,
sorting=ydnt,
]{biblatex}
% \bibliography{conferences,publications}
\bibliography{conferences,publications}

\defbibenvironment{bibliography}
  {\list
    {\printtext[labelnumberwidth]{% label format from numeric.bbx
        \printfield{labelprefix}%
        \printfield{labelnumber}}}
    {\setlength{\topsep}{0pt}% layout parameters from moderncvstyleclassic.sty
      \setlength{\labelwidth}{\hintscolumnwidth}%
      \setlength{\labelsep}{\separatorcolumnwidth}%
      \leftmargin\labelwidth%
      \advance\leftmargin\labelsep}%
    \sloppy\clubpenalty4000\widowpenalty4000}
  {\endlist}
  {\item}
\defbibenvironment{bibliographyconf}
  {\list
    {\printtext[labelnumberwidth]{% label format from numeric.bbx
        \printfield{month}%
        \setunit*{\space}%
        \printfield{year}}}
    {\setlength{\topsep}{0pt}% layout parameters from moderncvstyleclassic.sty
      \setlength{\labelwidth}{\hintscolumnwidth}%
      \setlength{\labelsep}{\separatorcolumnwidth}%
      \leftmargin\labelwidth%
      \advance\leftmargin\labelsep}%
    \sloppy\clubpenalty4000\widowpenalty4000}
  {\endlist}
  {\item}
\renewcommand{\mkbibbrackets}{}
% Remove'' `In:' from the references.
\renewbibmacro{in:}{}
% Replace 'and' with comma.
\renewcommand*{\finalnamedelim}{\addcomma\space}
\AtEveryBibitem{%
  \ifentrytype{inproceedings}
  {\clearfield{month}%
    \clearfield{year}}
  {}
}
\DefineBibliographyStrings{english}{
  january          = {Jan},
  february         = {Feb},
  march            = {Mar},
  april            = {Apr},
  may              = {May},
  june             = {Jun},
  july             = {Jul},
  august           = {Aug},
  september        = {Sep},
  october          = {Oct},
  november         = {Nov},
  december         = {Dec},
}

\newcommand\Colorhref[3][color1]{\href{#2}{\small\color{#1}#3}}


%-------------------------------------------------------------------------------
% Example usage of commands provided by the `moderncv` package:
% \cventry{year--year}{Degree/Job title}{Institute/Company}{Place}{Description}
% \cvline{left-column}{right-column}
% \cvlanguage{language 1}{Skill level}{Comment}
% Arguments not required can be left empty.


%-------------------------------------------------------------------------------
%	CURRICULUM VITAE
%-------------------------------------------------------------------------------
\begin{document}
% Print the CV title
\makecvtitle%


% ----------------------------------------------------------------------------------------
%	WORK EXPERIENCE SECTION
% ----------------------------------------------------------------------------------------

\section{Work Experience}

% \cventry{year--year}{Degree/Job title}{Institute/Company}{Place}{Grade}{Description}
\cventry{2019--present}{Researcher}{RWTH Aachen University, Aachen, Germany}{}{}{}
\cvline{}{Implement neural networks in Tensorflow 2 with custom training and loss functions}
\cventry
{2018--2019}
{Software developer}
{One Omega Seismics}
{Thuwal, Saudi Arabia}
{}
{}
\cvline{}{Refactor two geophysical numerical solvers in C and add integration tests}
\cventry{2010--2012}{Software developer}
{Fabrikant.ru}{Moscow, Russia}{}{}
\cvline{}{Develop a backend system for remote procedure calls in PHP}
\cventry{2008--2009}{Part-time software developer}
{AWAX Studio}{Moscow, Russia}{}{}
\cvline{}{Develop websites in PHP}


%-------------------------------------------------------------------------------
%	EDUCATION SECTION
%-------------------------------------------------------------------------------
\section{Education}

\cventry{2012--2018}
        {PhD in Applied Mathematics and Computational Science}{}{}{}{}
\cvline{}{King Abdullah University of Science and Technology, Saudi
Arabia}
% \cvline{\small dissertation}
%        {Numerical computation of linear stability of detonations}

\cventry
  {2004--2010}
  {Diploma (BSc + MSc), Engineer-Physicist}{}{}{}{}
\cvline{}{National Nuclear Research University MEPhI, Moscow, Russia}


% ------------------------------------------------------------------------------
\section{Languages}
\cvitem{}{English C2, Deutsch B1, Russian native}


% ------------------------------------------------------------------------------
\section{Skills}

\cvitem{}{Machine learning, numerical methods, linear algebra}
\cvitem{}{Unix, Python, NumPy, SciPy, Git, Bash, \LaTeX{} (9+ years)}
\cvitem{}{Tensorflow 2, Scikit-learn, Pandas (2+ years)}
\cvitem{}{C, C++, Cython, Matlab, Make, MPI, OpenMP, SQL (1+ year)}
\cvitem{}{Object-oriented programming}
\cvitem{}{Unit testing}
\cvitem{}{High-performant numerical codes}


%-------------------------------------------------------------------------------
\section{Hobbies}

\renewcommand{\listitemsymbol}{-~} % Changes the symbol used for lists

\cvitem{}{Playing and composing music, hiking, reading}

\end{document}
