%------------------------------------------------------------------------------
% Resume of Dmitry Kabanov.
% Created using LaTeX package `moderncv`.
% ------------------------------------------------------------------------------

%----------------------------------------------------------------------------------------
%	PACKAGES AND OTHER DOCUMENT CONFIGURATIONS
%----------------------------------------------------------------------------------------
% Font sizes: 10, 11, 12
% Paper: a4paper, letterpaper, a5paper, legalpaper, executivepaper, landscape
% Font: sans or roman
% Color: nocolor
% Draft/final option: draft, final
% Default options are: a4paper, 11pt, color, final
\documentclass[11pt,a4paper,sans]{moderncv} 

% CV theme: casual (default), classic, oldstyle, banking
% I use my own custom style `personal'.
\moderncvstyle{personal}
% CV color: blue (default), orange, green, red, purple, grey, black, burgundy
\moderncvcolor{blue}

% Make blue color from `moderncv` theme darker.
\definecolor{color1}{rgb}{0.05,0.22,0.61}
% Extra colors
\definecolor{white}{RGB}{255,255,255}
\definecolor{lightgray}{HTML}{B9B9B9}

% Width of the left column.
\setlength{\hintscolumnwidth}{0.3\textwidth}

% Font settings.
% Character encoding.
\usepackage[utf8]{inputenc}
% `Merriweather` is a modern screen-friendly font.
%\usepackage{merriweather}

% Page geometry.
\usepackage[scale=0.90]{geometry} % Reduce document margins
%\setlength{\hintscolumnwidth}{3cm} % Uncomment to change the width of the dates column
%\setlength{\makecvtitlenamewidth}{10cm} % For the 'classic' style, uncomment to adjust the width of the space allocated to your name
\recomputelengths

% Remove dots in the end of \cventry lines.
\usepackage{xpatch}
\xpatchcmd{\cventry}{.\strut}{\strut}{}{}

\newcommand\skill[1]{%
    \begin{tikzpicture}
        %\foreach [count=\i] \y in {#1}{
            %\draw[fill=maingray,maingray] (0,0) rectangle (8,\i+0.4);
            %\draw[fill=white,gray](0,\i) rectangle (\y,\i+0.4);
            %\node[above right] at (0,\i+0.4) {asdf};
        %}
        \draw[fill=lightgray, lightgray] (0, 0) rectangle (\linewidth, 0.32);
        \draw[fill=white, color1] (0, 0) rectangle (\linewidth*#1/5, 0.32);
    \end{tikzpicture}%
}

\makeatletter
\makeatother

% Personal data.
\title{R\'esum\'e}
% ------------------------------------------------------------------------------
%	NAME AND CONTACT INFORMATION SECTION
% ------------------------------------------------------------------------------

\firstname{Dmitry} % Your first name
\familyname{Kabanov} % Your last name

% All information in this block is optional, comment out any lines you don't need
% \title{Curriculum Vitae}

% \address{street and number}{postcode city}{country}
% optional, remove / comment the line if not wanted;
% the "postcode city" and "country" arguments can be omitted or provided empty.
\address{Aachen, Germany}{}{}

\phone[mobile]{+49~(1517)~087-66-15}
\email{kabanov.dmitry@gmail.com}
\social[linkedin]{dmitry-kabanov}
\social[github]{dmitry-kabanov}
% \photo[picture-height][frame-thickness]{photo}
% \photo[70pt][0.2pt]{photo} 

% \photo[picture-height][frame-thickness]{photo-filename}

%-------------------------------------------------------------------------------
% Example usage of commands provided by the `moderncv` package:
% \cventry{year--year}{Degree/Job title}{Institute/Company}{Place}{Description}
% \cvline{left-column}{right-column}
% \cvlanguage{language 1}{Skill level}{Comment}
% Arguments not required can be left empty.


%-------------------------------------------------------------------------------
%	CURRICULUM VITAE
%-------------------------------------------------------------------------------
\begin{document}
% Print the CV title
\makecvtitle%

%\section{Summary}
\cvline{}{%
Having passion for software development and mathematics, I am looking
for a~job that combines them such as \textbf{data scientist},
\textbf{machine learning engineer}, \textbf{researcher}.
I~have industrial experience (5 years) in~software development
as well as 9~years of~research experience.
I am dedicated to writing clean, well-tested, high-performant codes.
}

% ----------------------------------------------------------------------------------------
%	WORK EXPERIENCE SECTION
% ----------------------------------------------------------------------------------------

\section{Work Experience}

% \cventry{year--year}{Degree/Job title}{Institute/Company}{Place}{Grade}{Description}
\cventry{04/2019--present}{Researcher}{RWTH Aachen University}{Aachen, Germany}{}{}
\cvline{}{
  Work on three projects in scientific machine learning (supervised learning
  for physics-related tasks) that lead to two published papers,
  one in~preparation.
  Take leading role in code development.
  \begin{itemize}
  \item Implement neural networks in Tensorflow 2/Keras
        with custom training and loss functions
  \item Optimize hyperparameters via grid search and Pyperopt library
  \item Conduct computational experiments
  \item Write unit and integration tests using Pytest
  \end{itemize}
  \vspace*{-\baselineskip}\leavevmode
}

\cventry{07/2018--03/2019}{Software developer}{One Omega Seismics}{Thuwal, Saudi Arabia}{}{}
\cvline{}{
  I was hired by geophysical startup to improve quality of their seismic modeling software.
  \begin{itemize}
    \item Refactor two codes written in C programming language
    \item Add integration tests using Python and Bash
    \item Setup continuous integration (CI) on CircleCI
  \end{itemize}
  \vspace*{-\baselineskip}\leavevmode
}

\cventry{05/2010--09/2012}{Software developer}
{Fabrikant.ru}{Moscow, Russia}{}{}
\cvline{}{Develop a backend system for remote procedure calls in PHP}

\cventry{08/2008--12/2009}{Part-time software developer}
{AWAX Studio}{Moscow, Russia}{}{}
\cvline{}{Develop websites in PHP}


%-------------------------------------------------------------------------------
%	EDUCATION SECTION
%-------------------------------------------------------------------------------
\section{Education}

\cventry{09/2012--06/2018}
        {PhD in Applied Mathematics and Computational Science}{}{}{}{}
\cvline{}{\emph{King Abdullah University of Science and Technology}, Saudi
Arabia}
% \cvline{\small dissertation}
%        {Numerical computation of linear stability of detonations}

\cventry
  {09/2004--02/2010}
  {Diploma (BSc + MSc), Engineer-Physicist}{}{}{}{}
\cvline{}{\emph{National Nuclear Research University MEPhI}, Moscow, Russia}


% ------------------------------------------------------------------------------
\section{Languages}
\cvlanguage{English}{advanced (C1)}{9 years}
\cvlanguage{German}{upper intermediate (B2)}{3 years}
\cvlanguage{Russian}{native}{}

% ------------------------------------------------------------------------------
\section{Skills}
\cvitemwithcomment{\skill{5}}{%
  Unix, Python, NumPy, SciPy, Pytest, Git, Bash, \LaTeX{}}{9 years}
\cvitemwithcomment{\skill{4}}{%
  Machine learning, Tensorflow 2, Scikit-learn, Pandas}{2 years}
\cvitemwithcomment{\skill{2}}{%
  C, C++, Cython, Matlab, Make, MPI, OpenMP, SQL}{1 year}
\cvitemwithcomment{\skill{5}}{Object-oriented programming (OOP)}{10 years}
% \cvitem{}{Unit testing}
% \cvitem{}{High-performant numerical codes}


%-------------------------------------------------------------------------------
\section{Hobbies}

\renewcommand{\listitemsymbol}{-~} % Changes the symbol used for lists

\cvitem{}{Playing and composing music, hiking, reading}

\end{document}
